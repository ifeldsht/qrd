\documentclass{beamer}
 
\usepackage{graphicx}
\usepackage{qtree}
\usepackage[utf8]{inputenc}
\title{AI \& ML Introduction}
\author{I.F.}
\institute{mlschool2017.wordpress.com, repl.it}

\date{2018.01.06}
\begin{document}
 
\frame{\titlepage}
 
%%%%%%%%%%%%%%%%%%%%%%%%%%%%%%%%%%%%%%%%%%%%%%%%%%%%%%%%%%

\begin{frame}
\frametitle{The course plan}

\begin{enumerate}
\item Discrete decisions. Trees, search in trees.
\item Labyrinths, searching path -- different approaches.
\item Adding heuristics.
\item Weighted graphs, shortest paths.
\item Puzzles, 8-puzzle.
\item Two-player games, tic-tac-toe.
\item Tic-tac-toe: minimax, reinforcement learning.
\item Unknown environment, finding rules from observations.
\item Neural networks, finding rules of "Life".
\item Data analysis techniques.
\item Regression, approximations.
\item Forecast in time series.
\item Clustering, discrete states.
\item Decision trees.
\item What's next.
\end{enumerate}

\end{frame}

%%%%%%%%%%%%%%%%%%%%%%%%%%%%%%%%%%%%%%%%%%%%%%%%%%%%%%%%%%

\begin{frame}
\frametitle{Python}

\begin{enumerate}
\item python.org
\item Command line tools.
\item Basic types: numbers, string, boolean variables.
\item Lists, dictionaries.
\item Operations: loops, conditional statements.
\item Functions.
\item Modules.
\item File input/output
\item Useful libraries: scientific calculations, image processing, machine learning.
\item Useful resources
    \begin{enumerate}
        \item www.codecademy.com
        \item codingbat.com/python
        \item projecteuler.net
    \end{enumerate}
\end{enumerate}

\end{frame}

%%%%%%%%%%%%%%%%%%%%%%%%%%%%%%%%%%%%%%%%%%%%%%%%%%%%%%%%%%

\begin{frame}
\frametitle{Trees - 1}

\small{
\begin{figure}[H]
\centering
\Tree[ .{\textbf{Street}}  Cinema
[ .{\textbf{School}}  {Left stair}
  [ .{\textbf{Right stair}} {2nd floor}
    [ .{\textbf{3rd floor}} {\textbf{Math classroom}}
      Library {Chemical lab}
    ]
  ]
]
]
\caption{Path from street to classroom.}
\end{figure}
}
\end{frame}

%%%%%%%%%%%%%%%%%%%%%%%%%%%%%%%%%%%%%%%%%%%%%%%%%%%%%%%%%%

\begin{frame}
\frametitle{Trees - 2}

Navigating the tree:
\begin{figure}[H]
\centering
\Tree [ .1  [ .2 [ .4 7 ] [ .5 8 9 ] ]  [ .3 [ .6 10 ] ] ]
\end{figure}

$$1, 2, 3, 4, 5, 6, 7, 8, 9, 10 \ \ \textbf{or} \ \ 1, 2, 4, 7, 5, 8, 9, 3, 6, 10$$ 

The first path is called "in breadth", the second path is "in depth".

\textbf{A part of a tree is a tree}.
\begin{figure}[H]
\centering
\Tree [ .2 [ .4 7 ] [ .5 8 9 ] ]
\end{figure}

\end{frame}

\end{document}
