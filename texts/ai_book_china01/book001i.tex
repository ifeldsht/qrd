\chapter{Introduction and the course plan}

Resent developments in artificial intelligence and machine
learning quickly propagate into everyday life. It's hard
to find an area of human activity where autonomous agents
are not involved in one or the other way -- from banking and retail
to particle accelerators, medicine, government -- everywhere
we become deeper and deeper dependent upon an automated
decision making. This create a demand for better understanding of
basic principles behind machine learning techniques for a wide
range of professionals. 

This course is designed to give high school students an introduction
to the subject. We'll try to present important techniques
without going into technical details and to give a perspective
for future studies and developments.

The course starts with pretty traditional for introduction
to artificial intelligence topic of tree navigations. The
application to finding path in a maze is considered followed
by navigating graphs and finding short paths.

The next topic is about solving puzzles -- building decision trees,
reviewing ways to improve performance. After that we'll move
to two-player games and first learn how to build a rule-based
game engine for Tic-Tac-Toe. After reviewing limitations of the approach
we'll consider reinforcement learning -- the technique behind
recent achievements in Chess, Go and many other applications from
finance to automated translation -- to automatically build
a computer Tic-Tac-Toe player that can compete with a human. 

The last part of the course is an introduction to neural
networks. We'll not go into details of neural network training
as this would require deeper knowledge of mathematics and
optimization concepts. Instead we'll review the architecture 
of a perceptron and use available programming modules to reverse
engineer rules of Conway's Game of Life to give an idea how
autonomous agent may operate in an unknown environment.

The course include Appendix on Python programming. Students
not familiar with this programming languages will find
all need to complete the course.

The course contains three challenges. The work on them can be organized
as a competition between groups of students. 


