\chapter{Simple decisions}
\section{Finding your path}
We are starting with reviewing how you make everyday life decisions.
Let's say you want to find your class room and now you are on the street.
You can choose between two buildings: Cinema and School.

\begin{figure}[H]
\centering
\Tree [ .{\textbf{Street}}  Cinema  School ]
\end{figure}

You know that your classroom is in the school.
You have two choices and you enter the school building. Next question
you need to answer is which stair to choose -- on the left from the
entrance or on the right:

\begin{figure}[H]
\centering
\Tree[ .{\textbf{Street}}  Cinema
[ .{\textbf{School}}  {Left stair}  {Right stair} ]
]
\end{figure}

Let's say you go to the right and you and this is
your second choice. Now you need to choose the floor: 2nd or 3rd:

\begin{figure}[H]
\centering
\Tree[ .{\textbf{Street}}  Cinema
[ .{\textbf{School}}  {Left stair}
  [ .{\textbf{Right stair}} {2nd floor} {3rd floor} ]
]
]
\end{figure}

Your choice (3rd floor) brings you next set of options:
math classroom, library, chemical lab.
You choose math classroom and that is your destination.

\begin{figure}[H]
\centering
\Tree[ .{\textbf{Street}}  Cinema
[ .{\textbf{School}}  {Left stair}
  [ .{\textbf{Right stair}} {2nd floor}
    [ .{\textbf{3rd floor}} {\textbf{Math classroom}}
      Library {Chemical lab}
    ]
  ]
]
]
\caption{Path from street to classroom.}
\end{figure}

It's good to know your path in advance, but let's change the problem:
You don't know the path but you can identify the classroom
as soon as you are there. Now you have more options:

\begin{figure}[H]
\centering
\Tree[ .{Street} [ .Cinema  {Ticket \\office} ]
[ .{School}  [ .{Left stair} [ .{Basement} [ .Gym ] ] ]
  [ .{Right stair} [ .{2nd floor}  {Library} {Restroom} ]
    [ .{3rd floor} {Math classroom}
      Library {Chemical lab}
    ]
  ]
]
]
\caption{Adding choices to the path}
\end{figure}

In computer science (CS) the structure drawn on Figures 1.1 and 1.2
are called trees and
they play an important role in a number of applications.
The search in trees is one of central algorithms in building
intelligent agents that operate in known and unknown environments.

The assumption that each decision is made based
on a finite set of options (from the street you can enter school or cinema,
right stair brings you to the 2nd or 3rd floor) is a simplification,
but a powerful one. This allows,
at least in principle, to find a solution by reviewing all possible
combinations of decisions.

Our first task will be finding path to your classroom
given the tree from Figure 1.2.

\section{How to navigate a tree}

How do we navigate a tree? In other words - how do we go over
all nodes starting from the root? The answer depends on the desired order.
Let's consider the following tree:

\begin{figure}[H]
\centering
\Tree [ .1  [ .2 [ .4 7 ] [ .5 8 9 ] ]  [ .3 [ .6 10 ] ] ]
\caption{}
\end{figure}

The order of nodes can be
1, 2, 3, 4, 5, 6, 7, 8, 9, 10 or 1, 2, 4, 7, 5, 8, 9, 3, 6, 10 -- we either go layer by layer,
or left to right.
The first path is called "in breadth", the second path is "in depth".
Each path is associated with tree search algorithm
-- "depth first search (DFS)" and "breadth first search (BFS)".

We'll start with DFS and first make an obvious observation:
\textbf{a part of a tree is a tree}.
The part of the tree above is the tree:
\begin{figure}[H]
\centering
\Tree [ .2 [ .4 7 ] [ .5 8 9 ] ]
\end{figure}

This observation gives us an idea of a search.
Suppose we want to find a node
with a given name X (6 from Figure 1.3, "math classroom" from Figure 1.2).

\begin{leftborder}
\begin{enumerate}
\item We start with the root node.
\item We check the name - if the name is X we are done.
\item If it's not we are looking into the list of children.
\item If it's empty we are also done, search failed --
the tree doesn't have the node X.
\item If it's not empty we have have new tree with each children as a root.
\item For each children node perform search taking each children
node as a root for new smaller tree.
\end{enumerate}
\end{leftborder}

Let's see how it works for searching node 6 in the tree from Figure 1.3.

\begin{leftborder}
\begin{enumerate}
\item Root is 1.
\item It's not 6 and we continue with two trees - first with root node 2
and the second with root 3.
\item Node 2 is not 6. We continue with trees starting with nodes 4 and 5.
\item Node 4 is not 6 and we continue with tree starting with node 7.
\item Node 7 is not 6 and it doesn't have children.
\item We continue with root 5
\item Node 5 is not 6 and we continue with trees starting
with nodes 8 and 9.
\item Similar to node 7: 8,9 are not 6 and they don have children.
\item We continue with tree starting with node 3
\item Node 3 is not 6 and we continue with its child node which is 6.
\textbf{We are done!}
\end{enumerate}
\end{leftborder}


\section{How to describe a tree}

\begin{tcolorbox}
\textbf{Python:} 
You need to learn basic operations and collections in 
Python -- lists, dictionaries
\end{tcolorbox}

Let's see how we can
describe the tree from Figure 2. Each node has a name, for example "School"
and a list of children - ["Left stair", "Right stair"]. The list 
of children can be empty, in this case the node is called a leaf 
or a terminal node. The node without
a parent is called root node - in our case it's "Street".

Here is how we define
the upper part of the tree:

\medskip
\textbf{(Street, (Cinema,School))}
\medskip

Adding one more layer:

\medskip
\textbf{(Street, ((Cinema,(Ticket office)), (School,(Left stair, Right stair))))}
\medskip

It doesn't look convenient and transparent, adding one more layer
will make the description not readable. 
What would help is describing the tree incrementally - node by node:

Tree1 = (Left stair,[])

Tree2 = (Right stair,[]) 

Tree3 = (School,[Tree1,Tree2])

Here is what wee did: we created Tree1 as a pair of the node
name and the list of children (this list is empty, as for this example
stairs are terminal nodes -- they don't have children). 
The same is for Tree2. The list for Tree3 is not empty, it's constructed
from two previously defined trees -- we reuse trees to avoid
nested definitions. 

The remaining part of the description:

Tree4 = (Ticket office,[])

Tree5 = (Cinema,[Tree4])

Tree6 = (Street,[Tree3,Tree5])


Here is the full initialization for the tree from Figure 1.3 in Python:

\begin{lstlisting}[style=codelst,language=Python,caption={Python: tree description}]
Tree1 = ("Math classroom",[])
Tree2 = ("Library",[])
Tree3 = ("Chemical lab",[])
Tree4 = ("Library",[])
Tree5 = ("Restroom",[])
Tree6 = ("2nd floor",[Tree4,Tree5])
Tree7 = ("3rd floor",[Tree1,Tree2,Tree3])
Tree8 = ("Right stair",[Tree6,Tree7])
Tree9 = ("Basement",[])
Tree10 = ("Gym",[])
Tree11 = ("Left stair",[Tree9,Tree10])
Tree12 = ("School",[Tree8,Tree11])
Tree13 = ("Ticket office",[])
Tree14 = ("Cinema",[Tree13])
Tree15 = ("Street",[Tree12,Tree14])

print Tree15
\end{lstlisting}

The next step is to learn how to use this tree representation to navigate
a tree. Our first task is to go through all nodes and print their names.

Let's say we start from the root node. Our steps are:

\begin{leftborder}
\begin{enumerate}
\item print the name of the root node.
\item if the node doesn't have children we are done.
\item we know children nodes and we remember that each child is a root for own tree.
So we need to do 1-3) for all children.
\end{enumerate}
\end{leftborder}

\bigskip
\begin{tcolorbox}
\textbf{Assignment:} use steps 1-3) to manually reproduce 
steps of searching node 6 in
the Figure 1.3 
\end{tcolorbox}

\section{Recursion}

\begin{tcolorbox}
\textbf{Python:} 
You need to learn about functions, loops, and conditional statements.
\end{tcolorbox}

To solve our problem and print tree node names we need
to make one important observation: \textbf{a function can call itself}.
This is called \textbf{recursion} and is of particular importance in
CS, programming, linguistic. Before we go to the tree navigation,
let's look at the example traditionally used to demonstrate
the use of recursion: calculation of factorial. Factorial of a
positive integer number $n$
is denoted as $n!$ and defined as a product of all integers
from 1 to n:

$$n! = 1 \cdot 2 \cdot\dots\cdot n$$

For example:

$$3! = 1\cdot 2\cdot 3$$
$$5! = 1\cdot 2\cdot 3\cdot 4\cdot 5$$

Two observations:

\begin{leftborder}
\begin{enumerate}
\item $1! = 1$
\item $n! = n\cdot (n-1)!$ For example $5! = 5\cdot 4!$
\end{enumerate}
\end{leftborder}

Let's first imagine and then program the following function "factorial":

\begin{leftborder}
\begin{enumerate}
\item $factorial(1)$ returns $1$
\item $factorial(n)$ for $n\neq 1$ returns $n\cdot factorial(n-1)$
\end{enumerate}
\end{leftborder}

Here is the code for factorial calculation:
\begin{lstlisting}[style=codelst,language=Python,caption={Python: factorial}]
def factorial(n):
    if n == 1: return 1
    return n * factorial(n-1)

print factorial(5)
\end{lstlisting}

\bigskip
\begin{tcolorbox}
\textbf{Assignments:}
\begin{itemize}
\item Add printing function parameter to see the order of function calls
in the recursion.
\item Write function that calculates factorials, don't use recursion.
\item Write function that print first N Fibonacci numbers. 
Fibonacci numbers $F(i)$ are given by the following
formulas:

$$F(1) = 1$$ $$F(2) = 1$$ $$F(i) = F(i-1) + F(i-2) $$

First 6 numbers are: 1, 1, 2, 3, 5, 8.
Write two versions of the function: using loops and using recursion.
\end{itemize}
\end{tcolorbox}


\section{Navigating a tree (depth first)}
Now we can solve the problem of printing node names.
Here is the Python code --
it's the implementation of items 1-3) above.

\begin{lstlisting}[style=codelst,language=Python,caption={Python: printing node names (depth first)}]
def print_node_names(T):
    print T[0]
    for child in T[1]: print_node_names(child)
\end{lstlisting}

Let's look at these 3 lines of code in more details.

\begin{leftborder}
\begin{enumerate}
\item Here we are defining function of one parameter. The parameter is a tree.
\item The tree is represented by a pair and the first element T[0] of
the pair is the name. We print it.
\item The second element T[1] of the pair is the list of node's children. 
Line 3 goes over all children and calls
\lstinline{print_node_names} for each of them.
\end{enumerate}
\end{leftborder}

Now we apply this function to print all nodes of the tree we introduced above.
\bigskip
\begin{tcolorbox}
\textbf{Assignments:}
\begin{itemize}
\item Call the function \lstinline{print_node_names}
for the tree defined above. 
\item Change the function \lstinline{print_node_names}
to print nodes in the reverse order.
\end{itemize}
\end{tcolorbox}


