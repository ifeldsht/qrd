\chapter{Solving puzzles}

The problems we were solving so far had a static environment,
problems didn't change after our decisions -- mazes didn't
change after our moves, graphs didn't change when we were 
searching nodes. Real life environments where an autonomous 
agent is operating may react and change in the result of 
the agent's decisions. Games like Chess, Checkers, Go give
good models of changes after players moves.  

\section{8-Puzzle}

In this section we'll consider 8-Puzzle. The puzzle consists of
3x3 board and 8 tiles enumerated from 1 to 8. In the beginning 
all tiles are randomly placed on the board:

\begin{center}
\begin{tabular}{ | l | c | r | }
    \hline
    1 & 3 & 8 \\ \hline
    2 &   & 4 \\ \hline
    5 & 6 & 7 \\
    \hline
  \end{tabular}
\end{center}
Tiles can slide to the empty space. In the example above
a player can move tiles 2, 3, 4, or 6. The goal is to bring tiles
in the order:

\begin{center}
\begin{tabular}{ | l | c | r | }
    \hline
    1 & 2 & 3 \\ \hline
    4 & 5 & 6 \\ \hline
    7 & 8 &   \\
    \hline
  \end{tabular}
\end{center}
It's important to know that half of all randomly generated
initial tile positions correspond to non-solvable puzzles.
Any configuration where two tiles are swapped is not solvable 
(we'll leave this statement without a proof). For example

\begin{center}
\begin{tabular}{ | l | c | r | }
    \hline
    1 & 2 & 3 \\ \hline
    4 & 5 & 6 \\ \hline
    8 & 7 &   \\
    \hline
  \end{tabular}
\end{center}
(swapped 7 and 8) is not solvable. 

Before we start solving the puzzle we need to learn how to describe it.
As before with a maze we'll be using pair of indices (i,j) to refer to
a specific position:

\begin{center}
\begin{tabular}{ | l | c | r | }
    \hline
    (0,0) & (0,1) & (0,2) \\ \hline
    (1,0) & (1,1) & (1,2) \\ \hline
    (2,0) & (2,1) & (2,2) \\
    \hline
  \end{tabular}
\end{center}
First index in a pair identifies a row, second index - a column.
This representation is convenient for finding neighbors. For example
neighbors of (1,0) are (0,0), (1,1), and (2,0) - one element of
a pair is different from (1,0) by one. Note that we need to check
boundaries -- an index cannot be less than 0 and more than 2.

For other purposes this representation is less convenient and
a list or a tuple of 9 numbers may be preferable (we'll see how to
use two representations together soon). For example
$$
(1,3,2,5,7,4,0,8,6)
$$
corresponds to
\begin{center}
\begin{tabular}{ | l | c | r | }
    \hline
    1 & 3 & 2 \\ \hline
    5 & 7 & 4 \\ \hline
      & 8 & 6  \\
    \hline
  \end{tabular}
\end{center}
Note that we are using 0 to describe the empty space.

For the beginning we need several service functions:
\begin{enumerate}
\item We need a function that returns position indices for any number on
the board. For example for the board above we need a function
that returns (1,2) for 4.
\item We need a function that returns all possible moves for a given
board. For example for the board above we need a function that returns
5 and 8.
\item We need a function that returns updated board after a move.
For example move 5 (sliding tile 5 down to the empty space) for the 
board above produces new board $(1,3,2,0,7,4,5,8,6)$
\item We need to be able to compare two boards and return True for
identical boards -- we need to check if the current board is our
goal $(1,2,3,4,5,6,7,8,0)$
\item We need to be able to generate a random but solvable board. 
\end{enumerate}

\begin{tcolorbox}
\textbf{Python:}
You need to learn how to generate random numbers
in Python to continue.
\end{tcolorbox}

\section{Service functions}
1) Position of a particular tile \textbf{n} on a \textbf{board}. The
board is given as a list of 9 numbers:

\begin{lstlisting}[language=Python,style=codelst2,caption={Python: 8-puzzle, position of a tile}]
def position(board,n):
    # go over all rows ...
    for i in range(3):
        # ... and over all columns
        for j in range(3):
            if board[3*i+j] == n:
                return (i,j)
\end{lstlisting}
We just go row by row and column by column 
and return a pair of indices for the required tile.

2) All moves:

\begin{lstlisting}[language=Python,style=codelst2,caption={Python: 8-puzzle, all moves}]
def moves(board):
    # position of an empty cell
    i,j = position(board,0)
    all_moves = []
    # above the empty cell
    if i-1>=0: all_moves.append( board[(i-1)*3+j] )
    # below the empty cell
    if i+1<=2: all_moves.append( board[(i+1)*3+j] )
    # to the left of the empty cell
    if j-1>=0: all_moves.append( board[i*3+j-1] )
    # to the right of the empty cell
    if j+1<=2: all_moves.append( board[i*3+j+1] )
    return all_moves
\end{lstlisting}
Here we create a list of all neighbors of the empty cell -- above,
below, to the left, and to the right.

3) Board after a move. The move is given by a tile number.

\begin{lstlisting}[language=Python,style=codelst2,caption={Python: 8-puzzle, making a move, returning new board}]
from copy import copy

def move(board,n):
    # initializing new board as a copy of the board
    new_board = copy(board)
    # position of the empty cell
    i0, j0 = position(board,0)
    # position of the moving cell
    i, j = position(board,n)
    # swap: a,b=b,a
    new_board[i0*3+j0],new_board[i*3+j] = \
    new_board[i*3+j],new_board[i0*3+j0]
    return new_board
\end{lstlisting}
Note the new element -- in the first line we import function 
\textbf{copy} from the module \textbf{copy}. We need it to
create a copy of the original board element by element.

4) Comparing two boards

\begin{lstlisting}[language=Python,style=codelst2,caption={Python: 8-puzzle, comparing board}]
def cmp_boards(b1,b2):
    return tuple(b1)==tuple(b2)
\end{lstlisting}
Note that we need to compare two boards given by 
lists component by component.
This can be done either in a loop, or by converting lists into
tuples -- tuples are compared by components in Python.

%%%
\newpage

5) Generating a random solvable board.

\begin{lstlisting}[language=Python,style=codelst2,caption={Python: 8-puzzle, generating a board}]
from copy import copy
from random import random
from random import randint

def generate(board):
    new_board = copy(board)
    for _ in range(randint(10,20)):
        all_moves = moves(new_board)
        random_move_index = randint(0,len(all_moves)-1)
        random_move = all_moves[random_move_index]
        new_board = move(new_board,random_move)
    return new_board
\end{lstlisting}
We are starting with some board and making several randomly chosen 
moves. Here the number of moves is also random -- between 10 and 20.
In the loop we first find all possible moves for the current board.
Next we choose a random move from the list of all possible and 
making the move updating the board. If we start with a solvable 
board ([1,2,3,4,5,6,7,8,0]) we'll generate a solvable initial
board.

\begin{tcolorbox}
\textbf{Assignments:}
\begin{enumerate}
\item Write a function that prints a board. Use space instead of 0
for the empty space. For example for the board
given as a list [1,3,7,0,5,6,8,2,4] it prints
\begin{lstlisting}[language=bash]
137
 56
824
\end{lstlisting}
\item Write a function that generates and prints a solvable board.
\end{enumerate}
\end{tcolorbox}

\section{BFS}
Now as we have all service function, let's implement BFS for solving
the puzzle. The code is pretty similar to BFS we developed for
finding a path in a maze:

\begin{lstlisting}[language=Python,style=codelst2,caption={Python: 8-puzzle, BFS}]
def Puzzle8BFS(board,goal):
    Q = [([],board)]
    V = []
    while len(Q) > 0:
        path, cur = Q[0]
        Q = Q[1:]
        if cmp_boards(cur,goal): return (True,path)
        if tuple(Q) in V: continue
        V.append(tuple(cur))
        for m in moves(cur):
            Q.append((path+[m],move(cur,m)))
    return (False,[])
\end{lstlisting}
Similar to a maze problem we are creating a queue and a list of 
boards we already tried. The queue element is a pair: list of all moves
made before and the resulting board. While we have anything in the queue
we are taking its first element. If the board is our goal
we are returning True completion flag along with the 
list of moves we made to
solve the puzzle. If not and the board (\textbf{cur}) is in
the list of already visited we are skipping it. 
Otherwise we are continuing with the board and first we are appending
it to the list of visited. Next step is similar to the loop in
the BFS for mazes, only neighbors are substituted with all possible
moves -- we are appending pairs of moves and resulting boards
to the queue.

\begin{tcolorbox}
\textbf{Assignment:}
Combine all service functions with BFS function,
generate initial random board and solve the puzzle.
Experiment with the number of random moves used to generate 
initial board -- change (10,20) to (100,200) or any other
bigger numbers.
\end{tcolorbox}

You may find that the time needed to solve the puzzle can be
big. Let's add some measure to quantify time and memory
requirements -- number of steps needed to solve and maximum
size of the queue:

%%%
\newpage

\begin{lstlisting}[language=Python,style=codelst2,caption={Python: 8-puzzle, BFS with counters}]
def Puzzle8BFS(board):
    Q = [([],board)]
    V = []
    counter = 0
    max_size = 0
    while len(Q) > 0:
        path, cur = Q[0]
        Q = Q[1:]
        if cmp_boards(cur,goal): 
            return (True,counter,max_size,path)
        if tuple(Q) in V: continue
        V.append(tuple(cur))
        for m in moves(cur):
            Q.append((path+[m],move(cur,m)))
        if max_size < len(Q): max_size = len(Q)
        counter += 1
    return (False,counter,max_size,[])
\end{lstlisting}
Now we can see how different are numbers of steps and size of the 
queue for different initial boards.

\section{A*}
The number of moves and the resulting execution time
may be not satisfactory -- we need to improve the algorithm.

Similar to finding paths in mazes we need to rank possible moves --
order them based on some estimation of a move quality. For mazes
the estimation was based on the geometrical distance from the
current node to the goal. We need to introduce a distance 
from a current board to the goal.

Let's start with a simpler question -- how to introduce
the distance if we have only one tile and 8 empty spaces.
The tile is moved out of it's expected place:
\begin{center}
\begin{tabular}{ | l | c | r | }
    \hline
      &   &   \\ \hline
      &   &   \\ \hline
      & 1 &   \\
    \hline
  \end{tabular}
\end{center}
1 has to be in the top left position. We need to make two moves up 
and 1 move to the left. We can consider this number of moves from the
current position of a tile to the goal as a distance for a 
single tile. The sum of these distances for all tiles is the distance
for the board. Consider the following board:
 
\begin{center}
\begin{tabular}{ | l | c | r | }
    \hline
    1 & 3 & 8 \\ \hline
    2 &   & 4 \\ \hline
    5 & 6 & 7 \\
    \hline
  \end{tabular}
\end{center}
We need:

0 moves for the tile 1

1 move for the tile 2

1 move for the tile 3

2 moves for the tile 4

2 moves for the tile 5

2 moves for the tile 6

2 moves for the tile 7

3 moves for the tile 8

Total distance is $0+1+1+2+2+2+2+3=13$

Here is the code that implements the distance calculations

\begin{lstlisting}[language=Python,style=codelst2,caption={Python: 8-puzzle, distance between boards}]
def dist(b1,b2):
    total_dist = 0
    for i1 in range(3):
        for j1 in range(3):
            i2, j2 = position(b2,b1[i1*3+j1])
            total_dist += abs(i1-i2) + abs(j1-j2)
    return total_dist
\end{lstlisting}
We go in the loops over 3 rows and 3 columns.
For each cell ($i_1,j_1$) we have the tile number 
on the first board and can get the position ($i_2,j_2$) of that number on
the second board. Next we are increasing the total distance
by $|i_1-i_2|+|j_1+j_2|$. Note that the procedure counts also
for the empty cell.

Now as we have the notion of the distance from the current board
to the goal the next step should already be clear -- 
we can write the version of A* search algorithm.

%%%
\newpage

\begin{lstlisting}[language=Python,style=codelst2,caption={Python: 8-puzzle, A*}]
from heapq import heappop
from heapq import heappush

def Puzzle8AStar(board,goal):
    Q = []
    heappush(Q,(dist(board,goal),[],board))
    V = []
    counter = 0
    max_size = 0
    while len(Q) > 0:
        _, path, cur = heappop(Q)
        if cmp_boards(cur,goal): 
            return (True,counter,max_size,path)
        if tuple(Q) in V: continue
        V.append(tuple(cur))
        for m in moves(cur):
            m_board = move(cur,m)
            heappush(Q,(dist(m_board,goal),path+[m],m_board))
        if max_size < len(Q): max_size = len(Q)
        counter += 1
    return (False,counter,max_size,[])
\end{lstlisting}
As before we are using \textbf{heapq} module to work with
sorted elements. Each element of the queue has three
components now -- the distance to the goal (it's the first
element used for sorting), list of moves
executed from the original board, the board.

\begin{tcolorbox}
\textbf{Assignment:}
Combine all functions we've developed in this chapter,
generate initial boards and solve the puzzle with
BFS and A*. Compare the numbers of steps and maximum 
queue sizes in BFS and A*. Compare number of moves need to
solve the puzzle with BFS and A*.
\end{tcolorbox}


You may find that A* works faster and requires less memory,
but it may produce the puzzle solution with more moves
than BFS. A* doesn't guarantee the smallest number of moves,
but on average it works faster than BFS.

\section{Challenge}

Implement the N-puzzle -- adjust the code developed in this
chapter to make the size of the board a parameter. For example
15-puzzle (4x4 board):

\begin{center}
\begin{tabular}{ | l | c | c | r | }
    \hline
    1 & 2 & 3 & 4 \\ \hline
    5 & 6 & 7 & 8 \\ \hline
    9 & 10 & 11 & 12 \\ \hline
    13 & 14 & 15 & \\
    \hline
  \end{tabular}
\end{center}

Implement this puzzle on a non-square board, for example on the board
4x3:

\begin{center}
\begin{tabular}{ | l | c | c | r | }
    \hline
    1 & 2 & 3 \\ \hline
    4 & 5 & 6 \\ \hline
    7 & 8 & 9 \\ \hline
    10 & 11 &  \\
    \hline
  \end{tabular}
\end{center}

Try to introduce a different distance between the board and the goal.
In the chapter we've introduced the distance as a sum of distances
of all tiles. You may consider for example not a sum, but max distance.
You can come up with a generated initial board and compete for the
smallest number of steps of A* algorithm.
