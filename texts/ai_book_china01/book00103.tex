\chapter{Finding short paths}


\section{Weighted graphs}

As we already know graph is given by nodes and edges. One can think
about nodes as dots and about edges as lines connecting
a pair of nodes. Tree is a special case of graphs -- graph without loops.
We used different ways to name nodes -- words, when describing
path from street to classroom, (i,j) pair representing
position of a node in a maze. Going forward we'll be
enumerating nodes or assign them letters for compactness.

So far all edges where identical -- no weight was assigned
to an edge. At the same time there are applications where
weight can be introduced in a very natural way. If, for example,
nodes are cities connected by roads we can think about weights
as distances between cities. Graphs with weights assigned to
edges are called weighted graphs \footnote{We'll be assuming
all weights are positive.}. They appear in a number of
problems of logistics and planning when the cost of
each decision is different. Consider famous
"Travelling salesman problem" -- a salesman has to visit
several cities and he knows distances between them (not all
cities are connected by roads). He has to choose the shortest path.
In this case cities are nodes and edge weights are distances
between cities connected by roads.

Consider the following graph:




What is the shortest path from A to B?
Let's review possible paths:

