\chapter{Multi-player games}

In this chapter we'll consider decision-making in a multi-palyer
games -- the case when the state of a game can change unpredictably
due to a move of a counterparty. Games like Chess, Checkers, Go
are all good examples of two-player games. Building a computer
program that can play these games is difficult, but ideas behind
several approaches can be studied on a game as simple as Tic-Tac-Toe.

\section{Tic-Tac-Toe -- the game}

Two players make moves on a 3x3 board. They start with an empty board

\begin{center}
\begin{tabular}{ l | c | r }
     &  &  \\ \hline
     &  &  \\ \hline
     &  &  \\
  \end{tabular}
\end{center}
First player puts X in a free space,
the second player puts O. The player who first puts 3 symbols in a row is
the winner. Examples of winning positions:

\begin{center}
\begin{tabular}{ l | c | r }
    X & O & O \\ \hline
     & X &  \\ \hline
     &  & X  \\
  \end{tabular}
\end{center}

\begin{center}
\begin{tabular}{ l | c | r }
    X &  & X \\ \hline
    O & O & O \\ \hline
      & X & X  \\
  \end{tabular}
\end{center}

There may be conditions when the board is full and there is no winner (draw):

\begin{center}
\begin{tabular}{ l | c | r }
    X & O & X \\ \hline
    O & X & O \\ \hline
    O & X & X  \\
  \end{tabular}
\end{center}


We'll be describing the position on the board by the list of 9 symbols:
"X", "O", or "\ " (empty space). For example, three positions above will be
["X", "O", "O", "\ ", "X", "\ ", "\ ", "\ ", "X"],
["X", " ", "X", "O", "O", "O", "\ ", "X", "X"], and
["X", "O", "X", "O", "X", "O", "O", "X", "X"].

\\\\\\
\textbf{Python:} to continue you need to learn elements of
object-oriented programming -- classes in Python, constructors, member
functions and variables.
\\\\\\

There are several approaches to building a computer program that can
play Tic-Tac-Toe or other similar two-player games:

\begin{enumerate}
\item We can hard-code some strategy -- a set of rules the program
applies to choose next step. For example, two obvious rules are:
"\textbf{win}: if you have two symbols in a row and
can put the third symbol, do and win",
"\textbf{don't let win}: if on the next step your competitor
can win by putting third symbol, prevent this".
It can be shown that Tic-Tac-Toe has a winning strategy.
\item The first approach is hard to generalize -- it's close to impossible
to apply it to a game
just a little more complex than Tic-Tac-Toe. We can introduce
a general way of choosing next move that increases the chances of
a win -- so called minimax approach.
\item We'll find that minimax may require a lot of resources to evaluate
a move and the next approach will be the reinforcement learning
(Q-learning) to construct an evaluation function. The program
will first play with itself for some time and will eventually build a
model that can play against a human or other program.
\end{enumerate}

Let's start with a simple rule-based strategy -- we'll
implement two rules we've mentioned above (win and don't let win)
and for all other cases the program will be doing a random move.
We'll leave the devlopment of a stronger rule-based strategy for
the challenge in the end of the chapter.

We have to preapre several service elements:

\begin{enumetrate}
\item The class that implements the board and board-related functions.
\item Two calsses that implement a computer player and a human player.
Computer player is making a move based on one or the other algorithm, and
the human player enters moves from the console.
\item The class that implements the game -- the order of operations,
position evaluation, etc.
\end{enumerate}

All those entities -- boarad, players, game -- have own data and functionality
and it's convenient to organize it as class member variables and functions.

\subsection{The board}





