\chapter{Labyrinths}

Let's apply tree navigation techniques to an old problem
of finding path in a maze. For the beginning we'll be assuming that
maze is known (we have full description of a maze before we start),
and that maze doesn't contain loops.

\begin{labyrinth}{3}{4}
        \h -++
\v ++-- \h ---
\v ++++ \h --+
\v +--+ \h +++
\end{labyrinth}
\begin{labyrinth}{3}{4}
        \h -++
\v +-+- \h ---
\v ++++ \h ---
\v +--+ \h +++
\labyrinthsolution(0,1){uurddl}
\end{labyrinth}

Compare two mazes from the figure above. The top one doesn't have
loops. 

Our goal is to find a path in the maze from Entrance to Exit:

////////

The first question we need to answer is about the maze description.
The answer comes from the following observation: every time we
make a decision navigating the maze we make it from a finite set
of choices:
////////
When we are in location A we can move to location B, when in B we can move to C or D.
Let's enumerate all locations and list all possible moves:
////////
$$Entrance\rightarrow 1,\ 1\rightarrow 4,\ 
4\rightarrow 7,\ 7\rightarrow 8,\ 8\rightarrow 5\ or\ 9,
5\rightarrow 2,\ 2\rightarrow 3,\ 3\rightarrow 6\ or\ Exit.$$

This structure is know from the previous chapter. It's a tree:

\begin{figure}[H]
\centering
\Tree[ .Entrance [ .1 [ .4 [ .7 [ .8 [ [ .5 [ .2 [ 3. [ 6 Exit ] ] ] ] 9 ] ] ] ] ] ] 
\end{figure}

Assignment: describe this tree in Python and print the path from
Entrance to Exit.

The maze from the Figure 1?? is small and it's easy to describe it by
manually tracing all possible moves.
For bigger mazes we need an automated way of extracting trees
from schematic representations.
For the maze above this can be something like this:

0100000
0101111
0101010
0101000
0111110
0000000

Zeros represent walls and ones represent paths. You may find that the
size -- number of elements -- is different. This is the cost of introducing
walls -- they have a size on the scheme. Extracting a maze from an
image file (scanned or drawn) is technically possible, but difficult.

You can define the scheme of a maze inside your Python program, but
more flexible would be to save schemas as text files and have one
program that can work with different mazes. Now we need to learn how
to use files in Python.

\section{Python: input/output}

Simple type


