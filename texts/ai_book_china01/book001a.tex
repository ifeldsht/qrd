\chapter{Python basics}

\section{Programming -- basic concepts}

Let's start with reviewing elements that don't dependent
on the programming language and then see how they work in
Python. Programs implement algorithms and
as a rule take some data as an input and generate data as an output.
Data types are domain-specific - different applications require
different data elements and
data organization, but the most common types are:

\begin{leftborder}
\begin{enumerate}
\item Logical constants: True and False.
\item Single numbers: last price on the market, number of
cars on the parking, age, number of seconds between two events,
universal constants, etc. 5, 3.1415926, 123.97
\item Text: single characters, words, sentences, documents, text of news
\item Lists of elements: historical prices, weather records,
network traffic. Elements of lists can be accessed by their
position in the list - first element, second element, etc.
Lists are very common and fundamental in programming and in CS.
The name of on of eldest programming languages -- LISP --
stands for LISt Processor.
\end{enumerate}
\end{leftborder}

\section{A few words on programming environments}
If Python is not installed please go to python.org an follow
instructions for your operating system.
We assume a student can create a text file with the code of the program
and execute command:

\begin{lstlisting}[language=bash,style=codelst2,caption={Python: runnning program saved in filename.py}]
python filename.py
\end{lstlisting}

Traditionally the first program is printing "Hello world!".
\footnote{Files with source codes of all examples are attached.}
Here is its listing in Python:

\begin{lstlisting}[style=codelst2, language=Python,caption={Python: Hello World!}]
print "Hello world!"
\end{lstlisting}

\bigskip
\begin{tcolorbox}
\textbf{Assignment:} create file HelloWorld.py with this program
and run in the command line:
\begin{lstlisting}[language=bash,frame=single,caption={Python: run HelloWorld.py}]
python HelloWorld.py
\end{lstlisting}
\end{tcolorbox}

\section{Basic operations}

Start Python interactive shell and enter commands (Listing A.2) one by one.

\begin{lstlisting}[style=codelst,language=Python,caption={Basic types}]
# Boolean
b = True

# Integer number
x = 2

# Float number
y = 1.23

# String
s = "Abc"

# List of numbers:
L = [1,2,3,4,5]

# print everything:
print b, x, y, s, L

# Operations:
b2 = not b
print b, b2

x2 = x * 31
print x, x2

y2 = y*y
print y, y2

s2 = s.lower()
print s, s2

L2 = L[::-1]
print L, L2
\end{lstlisting}
Lines started with \# are comments, they are not executed.
Create file with commands and execute it. 

\section{Functions in Python}

We are using "def" to declare new function, list parameters in
parenthesis, and column ":" in the end of declaration
For example:

\begin{lstlisting}[style=codelst2,language=Python,caption={First function}]
def f(x): return x*x
\end{lstlisting}

Here we've declared function \textbf{f}. The function
 takes one parameter \textbf{x},
and returns the square of the parameter. The function
can fail if multiplication is not defined for the parameter \textbf{x} --
if it's a string for example. 

The function can have more than one line of code. We can re-write
the function from our example in the following way

\begin{lstlisting}[style=codelst2,language=Python,caption={Multi-line function}]
def f(x):
    x_squared = x*x
    return x_squared
\end{lstlisting}
Note indentation - usually it's 4 spaces. The indentation defines
the body of the function and affects "visibility" of variables. 
Consider the following code:

\begin{lstlisting}[style=codelst2,language=Python,caption={Visibility of variables}]
x = 1

def f(x):
    x2 = 2*x
    return x2

def g():
    x2 = 2*x
    return x2

print f(2)
print g()
\end{lstlisting}
First we've initialized variable \textbf{x} (line 1). Then we've
defined two functions:

\begin{enumerate}
\item \textbf{f} that takes a parameters with the same
name \textbf{x} and returns \textbf{2*x}. The variable x initialized on
the line 1 doesn't affect this function and
on the line 4 the variable \textbf{x} known to
the function f is equal 2.
\item \textbf{g} that doesn't take a parameter and returns \textbf{2*x} for
any \textbf{x} defined before the function call. In this case \textbf{g}
takes \textbf{x=1} defined on the line 1.
\end{enumerate}

Functions in Python can be nested - one function can be defined in
the body of other function:

\begin{lstlisting}[style=codelst2,language=Python,caption={Nested functions}]
x = 1

def f(x):
    x = 2*x
    def g(x):
        x = x+1
        return x
    return x + g(7)

print f(2)
\end{lstlisting}
There are three independent variables \textbf{x} here.
Note double indentation -- 8 spaces -- for the body of function \textbf{g}.

\bigskip
\begin{tcolorbox}
\textbf{Assignment:}
What are values of x when you are on the lines 1, 4, 6?
Add printing value of x in different parts of the code
to check your answer.
\end{tcolorbox}


\section{Conditions}
If you program need to change behaviour depending on some condition
you can use \textbf{if} statement.
Let's say you want to print numbers in words. On simple way of doing
this is to check the number and print accordingly.

\begin{lstlisting}[style=codelst2,language=Python,caption={Checking conditions}]
n = 1
if n = = 1:
    print "one"
elif n = = 2:
    print "two"
else:
    print "not one or two"
\end{lstlisting}
Here we are using two equality signs without spaces between 
to check the value.
\begin{itemize}
\item First we compare \textbf{n} with 1.
If n is equal to 1 we print the word "one".
\item If it's not we compare \textbf{n} with 2 (note the use of \textbf{elif} statement)
and print accordingly.
\item For all other cases (\textbf{n} is not 1 or 2) we use keyword \textbf{else}.
\end{itemize}


\section{Loops}
There exist cases when your program has to iterate over elements
of a list or a dictionary, or execute some part of a code several times
probably with different parameters. In such case we can use \textbf{for}
statement:

\begin{lstlisting}[style=codelst2,language=Python,caption="Loop"]
L = ["a", "b", "c", "def"]
for e in L:
    print e
\end{lstlisting}

Very common is the case when your program iterates over the
list elements using their indices. Index is position of an element
in the list:

\begin{lstlisting}[style=codelst2,language=Python,caption={Printing list elements by index - 1}]
L = ["a", "b", "c", "def"]
for i in range(4):
    print L[i]
\end{lstlisting}

\textbf{range(4)} is function returning list of integers from 0 to n-1.
The code above is equivalent to
\begin{lstlisting}[style=codelst2,language=Python,caption={Printing list elements by index - 2}]
L = ["a", "b", "c", "def"]
for i in [0,1,2,3]:
    print L[i]
\end{lstlisting}
Note that indices start from 0: index of the first element is 0, index of
the second element is 1, etc. For the list of N elements the index of the last element
is N-1.


\section{File input/output}

\subsection{Text files - reading/writing}

Text files of different formats -- not formated plain text,
comma-separated (CSV), JSON, XML -- are widely used for a number
of applications and we'll start with reading/writing text.

File is referenced by its name. File name may include directory name.
To start working with a file you need to open it:

\bigskip
\lstinline{f = open("filename.txt","r")}
\bigskip

Here function \textbf{open} takes two parameters - 
file name and mode of operation.
The mode \textbf{"r"} opens file for reading 
starting from the very beginning. Other modes are
\textbf{"w"} - open for writing. If file doesn't exist it'll 
be created. If file exists it'll be open as an empty one and 
existing content will be lost.
\textbf{"a"} - open for appending. If file exists it'll 
be open for writing to the end of the file.

The function \textbf{open} returns an object (variable \textbf{f}) 
used to actually read/write.

After reading/writing is done file must be closed:

\bigskip
\lstinline{f.close()}
\bigskip

Here is example:

\begin{lstlisting}[language=Python,style=codelst,caption={File operations}]
# create a file with one line of text inside:
f = open("filename.txt","w")
f.write("Hello world!")
f.close

# read the file and print its content:
f = open("filename.txt","r")
content = f.read()
f.close
print content

# append some text to the file
f = open("filename.txt","a")
f.write("Hello again")
f.close
\end{lstlisting}

\begin{tcolorbox}
\textbf{Assignments:}
experiment with opening files in different modes
and reading/writing. Be careful with modes of operation.
\end{tcolorbox}

\subsection{Command line arguments. Modules in Python.}

To open a file you need to provide a file name. This can be done
directly in the code (hardcode the name). In this case you'll need to change
your program to run for the other file. As an alternative you can
specify the name as a parameter in the command line:

\begin{lstlisting}[language=bash,frame=single]
python yourprogram.py filename.txt
\end{lstlisting}

The code above can be re-written in the following way:

\begin{lstlisting}[language=Python,style=codelst,caption={Command line parameter}]
import sys

# create a file with one line of text inside:
f = open(sys.argv[1],"w")
f.write("Hello world!")
f.close

# read the file and print its content:
f = open(sys.argv[1],"r")
content = f.read()
f.close
print content

# append some text to the file
f = open(sys.argv[1],"a")
f.write("Hello again")
f.close

\end{lstlisting}

First line contains new element: \textbf{import sys}.
Python is has a notion of a module -- the way of organizing
code. To start using a module it has to be imported.

Module \textbf{sys} provides system-level functionality.
\textbf{sys.argv} is the list of command line arguments starting
from the name of Python program file.

\begin{lstlisting}[language=Python,style=codelst2,caption={Python: print command line arguments}]
import sys

for arg in sys.argv:
    print arg
\end{lstlisting}

\begin{tcolorbox}
\textbf{Assignments:}
\begin{itemize}
\item Run the code above with different parameters.
\item Write Python program that prints itself.
\end{itemize}
\end{tcolorbox}

This code prints all parts of command line. Note that all 
parameters (elements of \textbf{sys.argv}) are strings.
To pass a numerical value you need to convert corresponding parameter:

\begin{lstlisting}[language=Python,style=codelst2,caption={Python: numbers in command line}]
import sys

x = sys.argv[1]
print x, type(x)
y = int(x)
print y, type(y)
\end{lstlisting}

Function \textbf{type} returns type of the argument --
string, integer, float, list, etc.


\section{File input/output, parsing strings}

Depending on the file format parsing its content 
may be more or less difficult.
For purposes of the course we need to parse maze descriptions 
saved in text files and convert them into trees.

Python program that reads and prints the content of the file:

\begin{lstlisting}[language=Python,style=codelst2,caption={Reading file and printing its content}]
import sys

f = open(sys.argv[1],"r")
file_content = f.read()
f.close()

print file_content
\end{lstlisting}

First we need to split the file content into lines. Lines in files
end with invisible end-of-line (EOL) symbol or two 
symbols depending on the
operating system convention. In most of the cases
the symbol is \textbf{$\sim n$} or two symbols \textbf{$\sim r\sim n$}.
Backslash in the beginning indicates a special
kind of symbols used to format the text. The following code:

\begin{lstlisting}[language=Python,style=codelst2,caption={Splitting file into strings}]
lines = file_content.split("\n")
print lines
\end{lstlisting}

\begin{tcolorbox}
\textbf{Assignment:}
Combine two pieces of code and confirm the type of 
variable \textbf{lines} (List).
Print it element by element.
\end{tcolorbox}

Next step is to convert each line into a 
list of logical constants (True, False)
so that True corresponds to a path and False 
corresponds to a wall. For example
convert \textbf{0101010} into 
\lstinline{[False,True,False,True,False,True,False]}. One way of doing
this is to go character by character in a loop over the entire string,
compare characters with 1 and 0, and append new logical value to a list:

\begin{lstlisting}[language=Python,style=codelst2,caption={String element by element}]
line = "0101010"
num = len(line)
L = []
for i in range(num):
    is_path = line[i] == "1"
    L.append(is_path)
\end{lstlisting}
In this piece of code we first use function \textbf{len} to calculate
number of characters in the line. Next we create an empty list \textbf{L}.
We go character by character using their indices in the string
and compare them with "1". Logical variable 
\textbf{is\_path} is initialized
depending on the result of comparison with "1" and then appended to 
the list.

This way of coding is valid, but not very efficient and compact.
Instead of using loop we can use so called list comprehensions. For
a list L the following expression

\bigskip
\lstinline{[f(x) for x in L]}
\bigskip
creates a list constructed by applying 
function \textbf{f} to each element
of the list. For example:

\begin{lstlisting}[language=Python,style=codelst,caption={Python: list comprehensions}]
def f1(x):
    return x*2

def f2(x):
    return x*x

L = [1,2,3,4]
L1 = [f1(x) for x in L]
L2 = [f2(x) for x in L]
print L, L1, L2

# or the same in place:
L1 = [x*2 for x in L]
L2 = [x*x for x in L]
print L, L1, L2
\end{lstlisting}

A string in Python can be treated as a list 
(with some limitations).
The following code converts string \textbf{line} into a
list of characters it contains:

\begin{lstlisting}[language=Python,style=codelst2,caption={String to list}
line = "abc"
L = [x for x in line]
print line, L
\end{lstlisting}

Now we can rewrite the code converting string element by element into
logical variables in the following way:

\begin{lstlisting}[language=Python,style=codelst2,caption={Python: string element by element}]
line = "0101010"
L = [x=="1" for x in line]
\end{lstlisting}
Here we use in place expression \textbf{x=="1"} 
for each element of the \textbf{line}.

\section{Image processing}

To proceed you need to install Python module Pillow. Run

\begin{lstlisting}[language=bash,frame=single]
pip install Pillow
\end{lstlisting}
or refer to online documentation (python-pillow.org) for your
operating system.

Here is the Python code that loads image from attached PNG file,
prints image size and mode,
converts image into list of pixels, prints maze definition,
creates new list of pixels RGB, changes color of several pixels
to red, and creates an image for the new list and saves it.

\begin{lstlisting}[language=Python,style=codelst,caption={Python: image operations}]
from PIL import Image

img = Image.open("maze_1s.png")

imglist = list(img.getdata())

w, h = img.size
print w, h

for i in range(h):
    row = ["0" if p>192 else "1" for p in imglist[(i*w):((i+1)*w)]]
    print "".join(row)

new_imglist = [(x,x,x) for x in imglist]
for i in range(100):
    new_imglist[i*w+i] = (255,0,0)

new_img = Image.new("RGB",img.size)
new_img.putdata(new_imglist)

new_img.save("test.jpg")
\end{lstlisting}

Let's review the code line by line:


